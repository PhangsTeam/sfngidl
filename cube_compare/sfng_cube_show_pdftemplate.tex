%\newcommand{\path}{../../../}     % ------------> to complete
%\newcommand{\path}{../../../ForLatex/}     % ------------> to complete

%---------------------------------PACKAGES----------------------------------
\documentclass[11pt]{article}
\usepackage{amsmath}
\usepackage{amssymb}
\usepackage{color} % new
\usepackage{float} % for figures position
\usepackage{rotating}
\usepackage{subfigure} %for multi figures
\usepackage{longtable}
\usepackage{subfigure}
\usepackage{lscape}

%---------------------------------------------------------------
\begin{document}

%-----------------------------TITLE DATA------------------------
\title{Simple PHANGS Cube Inspection}

%---------------------------MAKE FIRST PAGE--------------------
\maketitle

%\newpage
%\tableofcontents

%---------------------------------------------------------------------
%---------------------------DOCUMENT----------------------------------
%---------------------------------------------------------------------
\section{Notes}
\label{sect:notes}

%This is the template latex file that is used to generate a summary
%report showing the channel maps of a PHANGS spectral line cube. It
%must be located in the reports directory. It will be read by
%snfg_make_latex.pro and the figures and tables that are generated by
%phangs_cube_show.pro will be incorporated in order to generate the
%output summary pdf.

\section{Scope}
\label{sec:scope}

\noindent This document contains the channel maps of a spectral
line cube. It is automatically generated by the {\tt
  phangs\_cube\_show.pro}.

\section{Properties of Input Cube}
\label{sect:input_cube}
\noindent This table summarises the basic parameters of the input cube.
\input{inputcube_table}

\section{Channel Maps}
\label{sect:channel_maps}

\noindent The following figure shows the individual channels of the final cube.\\
\input{cube1_chanmaps_fig}

\end{document}
