%\newcommand{\path}{../../../}     % ------------> to complete
%\newcommand{\path}{../../../ForLatex/}     % ------------> to complete

%---------------------------------PACKAGES----------------------------------
\documentclass[11pt]{article}
\usepackage{amsmath}
\usepackage{amssymb}
\usepackage{color} % new
\usepackage{float} % for figures position
\usepackage{rotating}
\usepackage{subfigure} %for multi figures
\usepackage{longtable}
\usepackage{subfigure}
\usepackage{lscape}

%---------------------------------------------------------------
\begin{document}

%-----------------------------TITLE DATA------------------------
\title{PHANGS Cube Inspection}

%---------------------------MAKE FIRST PAGE--------------------
\maketitle

%\newpage
%\tableofcontents

%---------------------------------------------------------------------
%---------------------------DOCUMENT----------------------------------
%---------------------------------------------------------------------
\section{Notes}
\label{sect:notes}

%This is the template latex file that is used to generate a summary
%report for inspecting a PHANGS spectral line cubes. It must be located
%in the reports directory. It will be read by snfg_make_latex.pro and
%the figures and tables that are generated by phangs_cube_inspect.pro
%will be incorporated in order to generate the output summary pdf.

\section{Scope}
\label{sec:scope}

\noindent This document contains the results of inspecting a spectral
line cube. It is automatically generated by the {\tt
  sfng\_cube\_inspect.pro}.

\section{Properties of Input Cube}
\label{sect:input_cube}
\noindent This table summarises the basic parameters of the input cube.
\input{inputcube_table}

\section{Properties of Final Cube}
\label{sect:final_cube}
\noindent This table summarises the parameters of the final cube (after smoothing etc.)
\input{finalcube_table}

\section{Total Flux In Cube}
\label{sect:totflux}
\noindent  This table summarises the total flux in the final cube.
\input{flux_table}

\section{Flux Per Channel}
\label{sect:perchannel}
%\input{perchannel_table}

\noindent The following figures show the flux per channel in the final cube.
\input{flux_perchannel_fig}
\input{flux_perchannel_sm_fig}

\section{Noise Statistics}
\label{sec:noise_stats}

\noindent The table summarises the noise properties of the final
cube. The first column lists the RMS as determined by {\tt
  make\_noise\_cube.pro}, i.e. the average value of the pixels in the
'noise cube' named something like mygalaxy\_c1noise.fits. The second
column lists the RMS determined from pixel values in the 
signal-free region of the final cube. \\

\input{noise_table}

\noindent These figures show the RMS per channel.
\input{rms_perchannel_fig}
\input{rms_perchannel_nosm_fig}

\noindent These figures show the histogram of pixel values in the
 signal-free region of the final cube.
\input{c1_noisehisto_fig}


\section{Channel Maps}
\label{sect:channel_maps}

\noindent The following figure shows the individual channels of the final cube.\\
\input{cube1_chanmaps_fig}

%\noindent The following figure shows the individual channels of the model cube.\\
%\input{model1_chanmaps_fig}

%\noindent The following figure shows the individual channels of the residual cube.\\
%\input{residual1_chanmaps_fig}

%\noindent The following figure shows the individual channels of the clean support cube.\\
%\input{clean1_chanmaps_fig}

\section{Emission Mask}
\label{sect:emission_mask}

\noindent This table shows the number of good (x,y,v) pixels inside the significant emission mask.\\

\input{emission_mask_table}

\noindent The following figure shows the individual channels of the
significant emission mask (black=emission, white=no emission).\\

\input{emissionmask_chanmaps_fig}

\section{Power Spectra}
\label{sec:powerspec}
%\input{powerspec_table}
%\input{powerspec_figs}

\section{Processing Logs}
\label{sec:logs}
\input{processing_logs}

\end{document}
