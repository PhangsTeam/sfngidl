%\newcommand{\path}{../../../}     % ------------> to complete
%\newcommand{\path}{../../../ForLatex/}     % ------------> to complete

%---------------------------------PACKAGES----------------------------------
\documentclass[11pt]{article}
\usepackage{amsmath}
\usepackage{amssymb}
\usepackage{color} % new
\usepackage{float} % for figures position
\usepackage{rotating}
\usepackage{subfigure} %for multi figures
\usepackage{longtable}
\usepackage{subfigure}
\usepackage{lscape}

%---------------------------------------------------------------
\begin{document}

%-----------------------------TITLE DATA------------------------
\title{SFNG Cube Comparison}

%---------------------------MAKE FIRST PAGE--------------------
\maketitle

%\newpage
%\tableofcontents

%---------------------------------------------------------------------
%---------------------------DOCUMENT----------------------------------
%---------------------------------------------------------------------
\section{Notes}
\label{sect:notes}

%This is the template latex file that is used to generate a summary
%report for comparing two SFNG spectral line cubes. It must be located
%in the reports directory. It will be read by snfg_make_latex.pro and
%the figures and tables that are generated by sfng_cube_compare.pro
%will be incorporated in order to generate the output summary pdf.

\section{Scope}
\label{sec:scope}

\noindent  This document contains the results of comparing two spectral line
cubes. It is automatically generated by the {\tt
  sfng\_cube\_compare.pro}.

\section{Properties of Input Cubes}
\label{sect:input_cubes}
\noindent This table summarises the basic parameters of the input cubes.
\input{inputcube_table}

\section{Properties of Matched Comparison Cubes}
\label{sect:match_cubes}
\noindent This table summarises the parameters of the cubes after they have been brought to matched resolution, gridding, field-of-view, etc.
\input{matchcube_table}

\section{Total Flux In Cubes}
\label{sect:totflux}
\noindent  This table summarises the total flux in the matched cubes and the
difference cube. The second column restricts the comparison region to
where significant emission has been identified in both cubes.
\input{flux_table}

\section{Flux Per Channel}
\label{sect:perchannel}
%\input{perchannel_table}
These figures show the flux per channel in the matched cubes and the difference cube.
\input{flux_perchannel_fig}
\input{flux_perchannel_jsm_fig}
\input{diffcube_flux_perchannel_fig}
\input{diffcube_flux_perchannel_jsm_fig}
%\input{channelmap_figs}

\section{Noise Statistics}
\label{sec:noise_stats}

\noindent The table summarises the noise properties of the matched cubes. For the
input cubes, the first column lists the RMS as determined by {\tt
  make\_noise\_cube.pro}, i.e. the average value of the pixels in the
'noise cube' named something like mygalaxy\_c1noise.fits. The second
column lists the RMS determined from pixel values in the the joint
signal-free region of the matched cubes. For the difference cube, the
first column is determined from all pixels in the cube, the second
column is using pixels in the joint signal-free region only.\\

\input{noise_table}

\noindent These figures show the RMS per channel.

\input{rms_perchannel_fig}
\input{rms_perchannel_nosm_fig}

\noindent These figures show the histogram of pixel values in the
joint signal-free region of the matched cubes (i.e. only pixels
without emission in both cubes are shown).

\input{c1_noisehisto_fig}
\input{c2_noisehisto_fig}
\input{diffcube_noisehisto_fig}

\section{Correlation Metrics}
\label{sec:correlation_stats}

\noindent This table shows the correlation coefficients (Pearson,
Rank) and the slopes (Y on X, X on Y) of a linear correlation between
the pixel values in the matched cubes.

\input{correlation_table}

\noindent These figures illustrate the correlation between the pixel
values in the matched cubes.

\input{lincorr_c1c2_fig} \input{lindens_c1c2_fig}
\input{logcorr_c1c2_fig} \input{logdens_c1c2_fig}
\input{lincorr_c2c1_fig} \input{lindens_c2c1_fig}
\input{logcorr_c2c1_fig} \input{logdens_c2c1_fig}

\section{Channel Maps}
\label{sect:channel_maps}
%\input{emission_masks_table}

\section{Emission Masks}
\label{sect:emission_masks}
%\input{emission_masks_table}

\section{Image Fidelity}
\label{sec:fidelity}
%\input{fidelity_table}
These figures illustrate the overall fidelity CDF, determined over the region where emission is identified in both cubes. 
\input{fidelity_cube1_cdf_fig}
\input{fidelity_cube2_cdf_fig}

The following figure indicates the median fidelity in each channel, determined over the region where emission is identified in both cubes. 
\input{fidelity_perchannel_jsm_fig}

\section{Power Spectra}
\label{sec:powerspec}
%\input{powerspec_table}
%\input{powerspec_figs}

\section{Processing Logs}
\label{sec:logs}
\input{processing_logs}

\end{document}
